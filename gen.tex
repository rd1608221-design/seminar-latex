% Beamer presentation: Generative AI in Virtual Reality
\documentclass[11pt]{beamer}
\usetheme{Madrid}
\usecolortheme{dolphin}
\setbeamertemplate{navigation symbols}{}
\setbeamertemplate{footline}[frame number]
\usepackage{graphicx}
\usepackage{hyperref}
\title[Gen AI in VR]{Generative AI in Virtual Reality}
\author{Surya Sekhar Panda \\ Seminar: [Course / Dept]}
\institute{Sambalpur University Institute of Information and Technology}
\date{\today}
\begin{document}

\begin{frame}
  \titlepage
\end{frame}

\begin{frame}{Outline}
  \tableofcontents
\end{frame}

\section{Introduction}
\begin{frame}{What is Generative AI?}
  \begin{itemize}
    \item AI systems that \emph{create} new content (text, images, 3D models, audio, video).
    \item Examples: text generation, image synthesis, 3D shape generation, procedural content.
  \end{itemize}
\end{frame}

\begin{frame}{What is Virtual Reality (VR)?}
  \begin{itemize}
    \item Immersive, computer-generated 3D environments accessed via headsets/controllers.
    \item Key features: presence, interaction, and stereoscopic view.
  \end{itemize}
\end{frame}

\section{Why Put Gen AI in VR?}
\begin{frame}{Motivation}
  \begin{itemize}
    \item Reduce manual content creation (levels, assets, dialogue).
    \item Personalize experiences in real-time for each user.
    \item Enable adaptive storytelling and intelligent NPCs.
    \item Scale training and simulation scenarios dynamically.
  \end{itemize}
\end{frame}

\section{How It Works}
\begin{frame}{Core Components}
  \begin{itemize}
    \item \textbf{Content generation models}: 3D generators, image models, audio/text models.
    \item \textbf{Runtime integration}: Game engine (Unity/Unreal) + AI backend.
    \item \textbf{Interaction layer}: Speech-to-text, intent recognition, dialogue manager.
  \end{itemize}
\end{frame}

\begin{frame}{Typical Architecture}
  \begin{center}
    \includegraphics[width=0.7\textwidth]{ff.jpg}
  \end{center}
\end{frame}

\section{Tools and Libraries}
\begin{frame}{Recommended Tools}
  \begin{itemize}
    \item Game engines: Unity, Unreal Engine.
    \item 3D tools: Blender, MagicaVoxel.
    \item AI & services: Generative model APIs (text, image, 3D), speech APIs.
    \item Middleware: ROS (for robotics training), WebSockets / REST for realtime comms.
  \end{itemize}
\end{frame}

\section{Use Cases}
\begin{frame}{Key Applications}
  \begin{itemize}
    \item Education: adaptive labs and historical recreations.
    \item Healthcare: procedural patient simulations for training.
    \item Entertainment: personalized VR storytelling and open-world generation.
    \item Design & Architecture: instant walkthroughs of AI-generated concepts.
  \end{itemize}
\end{frame}

\section{Challenges \& Ethics}
\begin{frame}{Technical Challenges}
  \begin{itemize}
    \item Latency and real-time constraints.
    \item Asset quality and coherence across generated content.
    \item Model size, compute cost, and on-device limits.
  \end{itemize}
\end{frame}

\begin{frame}{Ethical Concerns}
  \begin{itemize}
    \item Deepfake risks and misinformation in immersive spaces.
    \item Privacy: voice data and behavioral tracking.
    \item Bias in generated content and representation issues.
  \end{itemize}
\end{frame}

\section{Conclusion}
\begin{frame}{Future Outlook}
  \begin{itemize}
    \item Better multimodal models will improve realism and interactivity.
    \item Edge inference and optimized models will reduce latency.
    \item Wider adoption in training, entertainment, and remote collaboration.
  \end{itemize}
\end{frame}

\begin{frame}{References \& Further Reading}
  \begin{itemize}
    \item Papers and articles on generative models, procedural content, and VR.
    \item (Add specific references or API docs you used.)
  \end{itemize}
\end{frame}

\begin{frame}{Thank you}
  \begin{center}
    Questions? \\
    \vspace{3mm}
    Contact: \href{mailto:youremail@example.com}{youremail@example.com}
  \end{center}
\end{frame}

\end{document}
